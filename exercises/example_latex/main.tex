\documentclass[12pt,a4paper]{article}
\usepackage[utf8]{inputenc}
\usepackage{amsmath}
\usepackage{amsfonts}
\usepackage{amssymb}
\usepackage{hyperref} % highlights the references
\usepackage{booktabs}
\usepackage{graphicx}

% for citations
\usepackage[style=authoryear]{biblatex}
\addbibresource{example_bib.bib}

\title{An Example \LaTeX\ Project}
\author{Ebenezer Scrooge\thanks{University of Southern Antarctica}}
\date{Version: \today}

\begin{document}
	\maketitle
	
	\begin{abstract}
		This is a rather long, yet nonetheless void-of-meaningfulness text, that rather beautifully fulfills its purpose as to divert the readers attention from more pressing issues such as actually writing stuff and not reading this piece of nonsense.
	\end{abstract}

	\tableofcontents
	\listoffigures
	\listoftables
	\newpage

\section{Introduction}
	
	As \textcite{Friedman2001} showed, data can be actually used. If the author of this written document want to use the parenthesis form of citation, you are in luck, this works as well \parencite{Friedman2001a}.
	
	If you want to get more into algo stuff, make sure to check out the book by \citeauthor{Gareth2013} which, properly cited, is displayed as \textcite{Gareth2013}.
	
	Always make sure your text is properly escaped like this \_ 100\%, 20\&.
	
\section{Tables}
	
	A table is a float, thus it might be moved somewhere else (especially if we have as little text as we currently have), but we can always reference a table using \verb|\autoref{KEY}| which results in \autoref{tab:first_table}.
	
	\begin{table}
		\caption{My First Table}
		\label{tab:first_table}
		\centering
		{\scriptsize This can be a paragraph describing the table.\par}
		\begin{tabular}{lcr}
			\hline 
			Name & Dir & Sales \\
			\hline
			Alice & C & \$12,000\\
			Bob & R & \$17,000\\
			\hline
		\end{tabular}
	\end{table}

	Or we can also reference a second table using \autoref{tab:input_table}.
	
	\begin{table}
		\caption{A second Table using input}
		\label{tab:input_table}
		\centering
		{\scriptsize This can be a paragraph describing the table.\par}
		\begin{tabular}{lcr}
	\hline 
	Name & Dir & Sales \\
	\hline
	Alice & C & \$12,000\\
	Bob & R & \$17,000\\
	\hline
\end{tabular}
	\end{table}

\section{Pictures}

	We can can address figures/pictures using autoref as well which results in \autoref{fig:homer}.
	
	\begin{figure}
		\centering
		\includegraphics[width=0.5\textwidth]{pictures/homer.png} \\
		\caption{Wonderful Homer}
		\label{fig:homer}
		\small Some more information if necessary.
	\end{figure}
	
	
	\newpage
	
	\printbibliography
	
\end{document}